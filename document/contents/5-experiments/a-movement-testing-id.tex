\subsection{Pengujian Gerakan}
\label{subsec:movementtesting}

\begin{figure} [ht]
  \centering
  \includegraphics[width=0.45\textwidth]{figures/rosgraph/simulation-movement-test.png}
  \IfLanguageName{english}{
    \caption{Node scheme of the movement testing on the real robot.}
  }{
    \caption{Skema \emph{node} pengujian gerakan di simulasi.}
  }
  \label{fig:rosgraphrealrobotmovementtest}
\end{figure}

\begin{figure} [ht]
  \centering
  \includegraphics[width=0.45\textwidth]{figures/rosgraph/real-robot-movement-test.png}
  \IfLanguageName{english}{
    \caption{Node scheme of the movement testing in the simulation.}
  }{
    \caption{Skema \emph{node} pengujian gerakan pada robot fisik.}
  }
  \label{fig:rosgraphsimulationmovementtest}
\end{figure}


Pengujian gerakan dilakukan dengan menjalankan \emph{node} \lstinline{move_for} sebagai \emph{node behavior} yang akan memerintahkan robot untuk bergerak dengan kecepatan tertentu selama kurun waktu tertentu.
Seperti yang terlihat pada Gambar \ref{fig:rosgraphsimulationmovementtest},
  di simulasi,
  \emph{node} \lstinline{move_for} akan terhubung dengan \emph{node} \lstinline{dienen_robot_navigation} untuk mengatur kecepatan dari robot yang ada di simulasi menggunakan \emph{topic} \lstinline{/navigation/maneuver_input}.
Sedangkan untuk pengujian di dunia nyata, seperti yang terlihat pada Gambar \ref{fig:rosgraphrealrobotmovementtest},
  peran dari \emph{node} \lstinline{dienen_robot_navigation} yang mengatur navigasi pada robot virtual akan digantikan oleh \emph{node} \lstinline{navigation} yang mengatur navigasi yang ada pada robot fisik.


\begin{figure}[ht]
  \centering
  \begin{tikzpicture}
    \begin{axis}[
        height=0.2\textwidth,
        width=0.45\textwidth,
        ylabel=\IfLanguageName{english}{Distance}{Jarak} (meter),xlabel=\IfLanguageName{english}{$K^{th}$ Attempt}{Percobaan Ke-$K$},
        legend style={
          at={(0.5,1.5)},
          anchor=north,
          legend columns=-1,
        },
        ymajorgrids,
        bar width=3pt,
        ybar=0pt,
        xmin=0.1,
        xmax=12.9,
        ymin=0,
        xtick distance=1,
        ytick distance=1,
      ]
      \addplot table[x=index,y=expecteddistance,col sep=comma]{data/linear-movement-sim.csv};
      \addplot table[x=index,y=measureddistance,col sep=comma]{data/linear-movement-sim.csv};
      \addplot table[x=index,y=odometrydistance,col sep=comma]{data/linear-movement-sim.csv};
      \IfLanguageName{english}{
        \legend{Estimated,Measurement,Odometry}
      }{
        \legend{Perkiraan,Pengukuran,Odometri}
      }
    \end{axis}
  \end{tikzpicture}
  \IfLanguageName{english}{
    \caption{Comparison of last distance results from the expected and measured values in the simulation.}
  }{
    \caption{Perbandingan hasil jarak akhir dari nilai yang diharapkan dan yang terukur di simulasi.}
  }
  \label{fig:linearmovementsim}
\end{figure}


\begin{figure}[ht]
  \centering
  \begin{tikzpicture}
    \begin{axis}[
        height=0.19\textwidth,
        width=0.45\textwidth,
        ylabel=\IfLanguageName{english}{Angle}{Sudut} (degree),xlabel=\IfLanguageName{english}{$K^{th}$ Attempt}{Percobaan Ke-$K$},
        legend style={
          at={(0.5,1.5)},
          anchor=north,
          legend columns=-1,
        },
        ymajorgrids,
        bar width=5pt,
        ybar=0pt,
        xmin=0.1,
        xmax=6.9,
        ymin=0,
        xtick distance=1,
        ytick distance=90,
      ]
      \addplot table[x=index,y=expected,col sep=comma]{data/angular-movement-sim.csv};
      \addplot table[x=index,y=measured,col sep=comma]{data/angular-movement-sim.csv};
      \addplot table[x=index,y=odometry,col sep=comma]{data/angular-movement-sim.csv};
      \IfLanguageName{english}{
        \legend{Estimated,Measurement,Odometry}
      }{
        \legend{Perkiraan,Pengukuran,Odometri}
      }
    \end{axis}
  \end{tikzpicture}
  \IfLanguageName{english}{
    \caption{Last orientation results of angular movement in the simulation.}
  }{
    \caption{Hasil orientasi akhir dari gerakan putar di simulasi.}
  }
  \label{fig:angularmovementsim}
\end{figure}


Pengujian gerakan terbagi menjadi dua bagian,
  pengujian gerakan linier dan pengujian gerakan sudut,
  masing-masing diujikan selama 10 detik pada robot virtual di lingkungan simulasi dan pada robot fisik di dunia nyata.
Seperti yang terlihat pada Gambar \ref{fig:linearmovementsim} dan Gambar \ref{fig:angularmovementsim},
  posisi dan orientasi odometri yang diterima robot memiliki nilai yang sama dengan posisi dan orientasi model robot di simulasi.
Hal ini terjadi karena nilai odometri yang dikirimkan oleh node \lstinline{dienen_robot_navigation} adalah nilai yang sama dengan transformasi model robot di simulasi.


\begin{figure}[ht]
  \centering
  \begin{tikzpicture}
    \begin{axis}[
        height=0.2\textwidth,
        width=0.45\textwidth,
        ylabel=\IfLanguageName{english}{Distance}{Jarak} (meter),xlabel=\IfLanguageName{english}{$K^{th}$ Attempt}{Percobaan Ke-$K$},
        legend style={
          at={(0.5,1.5)},
          anchor=north,
          legend columns=-1,
        },
        ymajorgrids,
        bar width=3pt,
        ybar=0pt,
        xmin=0.1,
        xmax=12.9,
        ymin=0,
        xtick distance=1,
        ytick distance=1,
      ]
      \addplot table[x=index,y=expecteddistance,col sep=comma]{data/linear-movement-real.csv};
      \addplot table[x=index,y=measureddistance,col sep=comma]{data/linear-movement-real.csv};
      \addplot table[x=index,y=odometrydistance,col sep=comma]{data/linear-movement-real.csv};
      \IfLanguageName{english}{
        \legend{Estimated,Measurement,Odometry}
      }{
        \legend{Perkiraan,Pengukuran,Odometri}
      }
    \end{axis}
  \end{tikzpicture}
  \IfLanguageName{english}{
    \caption{Last distance results of linear movement on the real robot.}
  }{
    \caption{Hasil jarak akhir dari gerakan linier pada robot fisik.}
  }
  \label{fig:linearmovementreal}
\end{figure}


\begin{figure}[ht]
  \centering
  \begin{tikzpicture}
    \begin{axis}[
        height=0.2\textwidth,
        width=0.45\textwidth,
        ylabel=\IfLanguageName{english}{Angle}{Sudut} (degree),
        xlabel=\IfLanguageName{english}{$K^{th}$ Attempt}{Percobaan Ke-$K$},
        legend style={
          at={(0.5,1.5)},
          anchor=north,
          legend columns=-1,
        },
        ymajorgrids,
        bar width=5pt,
        ybar=0pt,
        xmin=0.1,
        xmax=6.9,
        ymin=0,
        xtick distance=1,
        ytick distance=90,
      ]
      \addplot table[x=index,y=expected,col sep=comma]{data/angular-movement-real.csv};
      \addplot table[x=index,y=measured,col sep=comma]{data/angular-movement-real.csv};
      \addplot table[x=index,y=odometry,col sep=comma]{data/angular-movement-real.csv};
      \IfLanguageName{english}{
        \legend{Estimated,Measurement,Odometry}
      }{
        \legend{Perkiraan,Pengukuran,Odometri}
      }
    \end{axis}
  \end{tikzpicture}
  \IfLanguageName{english}{
    \caption{Comparison of last orientation results from the expected and measured values on the real robot.}
  }{
    \caption{Perbandingan hasil orientasi akhir dari nilai yang diharapkan dan yang terukur pada robot fisik.}
  }
  \label{fig:angularmovementreal}
\end{figure}


Berbeda dengan pengujian di lingkungan simulasi,
  pengujian di dunia nyata memiliki hasil yang berbeda antara pengukuran langsung dengan nilai odometri yang diterima oleh robot.
Seperti yang terlihat pada Gambar \ref{fig:linearmovementreal} dan Gambar \ref{fig:angularmovementreal},
  terdapat perbedaan antara keduanya yang disebabkan oleh ketidakakuratan yang ada pada sensor yang digunakan untuk mendapatkan nilai posisi dan orientasi.
Walaupun begitu,
  terlepas dari adanya perbedaan tingkat akurasi, dengan menggunakan \emph{node behavior} yang sama,
  robot tergolong mampu melakukan perintah gerakan yang sesuai ketika diujikan di simulasi maupun di dunia nyata.
