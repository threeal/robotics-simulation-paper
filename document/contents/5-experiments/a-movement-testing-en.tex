\subsection{Movement Testing}
\label{subsec:movementtesting}

\begin{figure} [ht]
  \centering
  \includegraphics[width=0.45\textwidth]{figures/rosgraph/movement-sim.png}
  \IfLanguageName{english}{
    \caption{Node scheme of the movement testing in the simulation.}
  }{
    \caption{Skema \emph{node} pengujian gerakan di simulasi.}
  }
  \label{fig:rosgraphmovementsim}
\end{figure}

\begin{figure} [ht]
  \centering
  \includegraphics[width=0.45\textwidth]{figures/rosgraph/movement-real.png}
  \IfLanguageName{english}{
    \caption{Node scheme of the movement testing on the real robot.}
  }{
    \caption{Skema \emph{node} dari pengujian gerakan pada robot fisik.}
  }
  \label{fig:rosgraphmovementreal}
\end{figure}


Movement testing is done by running the \lstinline{move_for} node as a behavior node that will instruct the robot to move at a certain speed for a certain period of time.
As shown in figure \ref{fig:rosgraphmovementsim},
  in the simulation,
  the \lstinline{move_for} node will be connected to the \lstinline{navigation_plugin} node to control the speed of the robot using the \lstinline{/navigation/maneuver_input} topic.
As for testing in the real world,
  as shown in figure \ref{fig:rosgraphmovementreal},
  the role of the \lstinline{navigation_plugin} node which manages the navigation on the virtual robot will be replaced by the \lstinline{navigation} node which manages the navigation on the real robot.


\begin{figure}[ht]
  \centering
  \begin{tikzpicture}
    \begin{axis}[
        height=0.2\textwidth,
        width=0.45\textwidth,
        ylabel=\IfLanguageName{english}{Distance}{Jarak} (meter),xlabel=\IfLanguageName{english}{$K^{th}$ Attempt}{Percobaan Ke-$K$},
        legend style={
          at={(0.5,1.5)},
          anchor=north,
          legend columns=-1,
        },
        ymajorgrids,
        bar width=3pt,
        ybar=0pt,
        xmin=0.1,
        xmax=12.9,
        ymin=0,
        xtick distance=1,
        ytick distance=1,
      ]
      \addplot table[x=index,y=expecteddistance,col sep=comma]{data/linear-movement-sim.csv};
      \addplot table[x=index,y=measureddistance,col sep=comma]{data/linear-movement-sim.csv};
      \addplot table[x=index,y=odometrydistance,col sep=comma]{data/linear-movement-sim.csv};
      \IfLanguageName{english}{
        \legend{Estimated,Measurement,Odometry}
      }{
        \legend{Perkiraan,Pengukuran,Odometri}
      }
    \end{axis}
  \end{tikzpicture}
  \IfLanguageName{english}{
    \caption{Comparison of last distance results from the expected and measured values in the simulation.}
  }{
    \caption{Perbandingan hasil jarak akhir dari nilai yang diharapkan dan yang terukur di simulasi.}
  }
  \label{fig:linearmovementsim}
\end{figure}


\begin{figure}[ht]
  \centering
  \begin{tikzpicture}
    \begin{axis}[
        height=0.19\textwidth,
        width=0.45\textwidth,
        ylabel=\IfLanguageName{english}{Angle}{Sudut} (degree),xlabel=\IfLanguageName{english}{$K^{th}$ Attempt}{Percobaan Ke-$K$},
        legend style={
          at={(0.5,1.5)},
          anchor=north,
          legend columns=-1,
        },
        ymajorgrids,
        bar width=5pt,
        ybar=0pt,
        xmin=0.1,
        xmax=6.9,
        ymin=0,
        xtick distance=1,
        ytick distance=90,
      ]
      \addplot table[x=index,y=expected,col sep=comma]{data/angular-movement-sim.csv};
      \addplot table[x=index,y=measured,col sep=comma]{data/angular-movement-sim.csv};
      \addplot table[x=index,y=odometry,col sep=comma]{data/angular-movement-sim.csv};
      \IfLanguageName{english}{
        \legend{Estimated,Measurement,Odometry}
      }{
        \legend{Perkiraan,Pengukuran,Odometri}
      }
    \end{axis}
  \end{tikzpicture}
  \IfLanguageName{english}{
    \caption{Last orientation results of angular movement in the simulation.}
  }{
    \caption{Hasil orientasi akhir dari gerakan putar di simulasi.}
  }
  \label{fig:angularmovementsim}
\end{figure}


The movement test is divided into two parts,
  linear movement testing and angular movement testing,
  each is tested with various combinations of speed values for 3 seconds on a virtual robot in a simulation environment and on a real robot in the real world.
As shown in figure \ref{fig:linearmovementsim} and figure \ref{fig:angularmovementsim},
  the last distance and orientation of the odometry received by the robot relatively has the same value as the last distance and orientation of the estimated value from the given speed and the measurement from the starting point of the robot model in the simulation.
From these results,
  the error percentage of the odometry value compared to the average of estimated and the measurement value in the simulation is 2.6\%.


\begin{figure}[ht]
  \centering
  \begin{tikzpicture}
    \begin{axis}[
        height=0.2\textwidth,
        width=0.45\textwidth,
        ylabel=\IfLanguageName{english}{Distance}{Jarak} (meter),xlabel=\IfLanguageName{english}{$K^{th}$ Attempt}{Percobaan Ke-$K$},
        legend style={
          at={(0.5,1.5)},
          anchor=north,
          legend columns=-1,
        },
        ymajorgrids,
        bar width=3pt,
        ybar=0pt,
        xmin=0.1,
        xmax=12.9,
        ymin=0,
        xtick distance=1,
        ytick distance=1,
      ]
      \addplot table[x=index,y=expecteddistance,col sep=comma]{data/linear-movement-real.csv};
      \addplot table[x=index,y=measureddistance,col sep=comma]{data/linear-movement-real.csv};
      \addplot table[x=index,y=odometrydistance,col sep=comma]{data/linear-movement-real.csv};
      \IfLanguageName{english}{
        \legend{Estimated,Measurement,Odometry}
      }{
        \legend{Perkiraan,Pengukuran,Odometri}
      }
    \end{axis}
  \end{tikzpicture}
  \IfLanguageName{english}{
    \caption{Last distance results of linear movement on the real robot.}
  }{
    \caption{Hasil jarak akhir dari gerakan linier pada robot fisik.}
  }
  \label{fig:linearmovementreal}
\end{figure}


\begin{figure}[ht]
  \centering
  \begin{tikzpicture}
    \begin{axis}[
        height=0.2\textwidth,
        width=0.45\textwidth,
        ylabel=\IfLanguageName{english}{Angle}{Sudut} (degree),
        xlabel=\IfLanguageName{english}{$K^{th}$ Attempt}{Percobaan Ke-$K$},
        legend style={
          at={(0.5,1.5)},
          anchor=north,
          legend columns=-1,
        },
        ymajorgrids,
        bar width=5pt,
        ybar=0pt,
        xmin=0.1,
        xmax=6.9,
        ymin=0,
        xtick distance=1,
        ytick distance=90,
      ]
      \addplot table[x=index,y=expected,col sep=comma]{data/angular-movement-real.csv};
      \addplot table[x=index,y=measured,col sep=comma]{data/angular-movement-real.csv};
      \addplot table[x=index,y=odometry,col sep=comma]{data/angular-movement-real.csv};
      \IfLanguageName{english}{
        \legend{Estimated,Measurement,Odometry}
      }{
        \legend{Perkiraan,Pengukuran,Odometri}
      }
    \end{axis}
  \end{tikzpicture}
  \IfLanguageName{english}{
    \caption{Comparison of last orientation results from the expected and measured values on the real robot.}
  }{
    \caption{Perbandingan hasil orientasi akhir dari nilai yang diharapkan dan yang terukur pada robot fisik.}
  }
  \label{fig:angularmovementreal}
\end{figure}


In contrast to the testing results in the simulation environment,
  movement testing results in the real world have a little difference between the estimated and measurement value and the odometry values received by the robot.
As shown in figure \ref{fig:linearmovementreal} and figure \ref{fig:angularmovementreal},
  there is a difference between the two value due to combinations of several factors like inaccuracies in the sensors and the occurance of slip when the robot move with high acceleration.
Because of these factors,
  the error percentage of the odometry value compared to the average of estimated and the measurement value on the real robot is 12.5\%.
However,
  despite the differences in the level of accuracy,
  by using the same node behavior,
  the robot is capable of carrying out appropriate movement commands when tested in the simulation and the real world.
