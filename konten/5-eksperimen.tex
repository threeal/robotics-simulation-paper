\section{Eksperimen}
\label{sec:eksperimen}

\subsection{Pengujian Gerakan}

\begin{figure} [ht]
  \centering
  \includegraphics[scale=0.25]{gambar/nodeujigerak.png}
  \caption{Skema node pengujian gerakan linier.}
  \label{fig:nodeujigerak}
\end{figure}

Pengujian gerakan dilakukan dengan menjalankan node \lstinline{move_for} sebagai \emph{node behavior} yang akan memerintahkan robot untuk bergerak dengan kecepatan tertentu selama kurun waktu tertentu.
Seperti yang terlihat pada Gambar \ref{fig:nodeujigerak}, di simulasi, node \lstinline{move_for} akan terhubung dengan node \lstinline{dienen_robot_navigation} untuk mengatur kecepatan dari robot yang ada di simulasi melalui topic \lstinline{/navigation/maneuverinput}.
Sedangkan untuk pengujian pada robot fisik, peran dari node \lstinline{dienen_robot_navigation} yang mengatur navigasi pada robot virtual akan digantikan oleh node \lstinline{navigation} yang mengatur navigasi yang ada pada robot fisik.

\begin{table}
  \caption{Hasil pengujian gerakan linier di simulasi selama 10 detik.}
  \label{tab:hasilujiliniersimulasi}
  \centering
  \begin{tabular}{rr|rr|rr}
    \toprule
    \multicolumn{2}{c|}{Kecepatan} &
    \multicolumn{2}{|c|}{Posisi di Simulasi} &
    \multicolumn{2}{|c}{Posisi Odometri} \\
    \midrule
    x (m/min) & y (m/min) & x (m) & y (m) & x(m)  & y(m) \\
    \midrule
    40        & 0         & 5.2   & 0.0   & 5.2   & 0.0 \\
    60        & 0         & 7.9   & 0.0   & 7.9   & 0.0 \\
    -40       & 0         & -5.2  & 0.0   & -5.2  & 0.0 \\
    0         & 40        & 0.0   & 5.4   & 0.0   & 5.4 \\
    0         & -40       & 0.0   & -5.3  & 0.0   & -5.3 \\
    40        & 20        & 5.0   & 3.0   & 5.0   & 3.0 \\
    -20       & 40        & -2.9  & 5.5   & -2.9  & 5.5 \\
    \bottomrule
  \end{tabular}
\end{table}

\begin{table}
  \caption{Hasil pengujian gerakan angular di simulasi selama 10 detik.}
  \label{tab:hasilujiangularsimulasi}
  \centering
  \begin{tabular}{r|r|r}
    \toprule
    Kecepatan   & Orientasi di Simulasi & Orientasi Odometri \\
    \midrule
    z (rad/min) & z (deg)               & z (deg) \\
    \midrule
    40          & 39.0                  & 39.0 \\
    120         & 118.7                 & 118.7 \\
    -40         & -39.2                 & -39.2 \\
    -120        & -117.1                & -117.1 \\
    \bottomrule
  \end{tabular}
\end{table}

Pengujian gerakan terbagi menjadi dua bagian, pengujian gerakan linier dan pengujian gerakan angular, masing-masing diujikan pada robot virtual di lingkungan simulasi dan pada robot fisik di dunia nyata.
Seperti yang terlihat pada Tabel \ref{tab:hasilujiliniersimulasi} dan Tabel \ref{tab:hasilujiangularsimulasi}, posisi dan orientasi odometri yang diterima robot memiliki nilai yang sama dengan posisi dan orientasi model robot di simulasi.
Hal ini terjadi karena nilai odometri yang dikirimkan oleh node \lstinline{dienen_robot_navigation} adalah nilai yang sama dengan transformasi model robot di simulasi.
